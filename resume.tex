
\documentclass[margin,line]{resume}

\begin{document}
\name{\Large Paula Gombar}
\begin{resume}

\section{\mysidestyle Personal Information}

\begin{tabular}{@{}ll@{}}
% Address: & Brace Domany 6, 10 000 Zagreb, Croatia \\
Contact: & gombarica@gmail.com, pgombar.github.io \\
Languages: & Croatian (native), English (level C1), Spanish (level B2) \\
\end{tabular}

%\vspace{1mm}
\section{\mysidestyle Education}

\textbf{Radboud University}, Nijmegen, the Netherlands \vspace{1mm}\\\vspace{1mm}%
\textsl{Master Specialization in Data Science} \hfill January 2017 -- July 2017\\%
Recipient of the Erasmus scholarship to study abroad for a semester. Focusing on Data Science with courses such as: Machine Learning in Practice, Research Seminar in Data Science, and Cognitive Computational Modeling of Language and Web Interaction. \\
\vspace{-3mm}\hspace{-1mm}\hfill%

\vspace{-2mm}
\textbf{University of Zagreb}, Zagreb, Croatia \vspace{1mm}\\\vspace{1mm}%
\textsl{M.Sc \& B.Sc.\ in Computer Science} \hfill September 2012 -- July 2018\\%
Master Thesis supervised by Berkant Barla Cambazoglu: \textit{Recency Ranking Models for Web Search} \\
Bachelor Thesis supervised by Jan \v{S}najder: \textit{Contextual Sentiment Analysis of Croatian Expressions}
%\vspace{-3mm}%\hspace{-1mm}\hfill%

\vspace{-1mm}
\section{\mysidestyle Work Experience}

\textbf{Microsoft, Cloud \& Enterprise}, Vancouver, British Columbia, Canada \vspace{1mm}\\\vspace{1mm}%
\textsl{Software Engineer} \hfill November 2018 -- current\\%
Working on Linux Guest Agent in Azure Compute. Linux Guest Agent is a secure, lightweight process that manages virtual machine (VM) interaction with the Azure Fabric Controller. The Guest Agent is responsible for many functional aspects of deploying and managing Azure VMs, including running VM extensions. \\
Technologies used: Linux, Python, Bash, Azure, Git.

\textbf{NTENT}, Barcelona, Spain \vspace{1mm}\\\vspace{1mm}%
\textsl{Data Science Intern} \hfill March 2018 -- July 2018\\%
Under the supervision of Berkant Barla Cambazoglu, carried out a research project on recency ranking, \textit{Recency Ranking Models for Web Search}, ultimately my Master Thesis. Devised, implemented and fully integrated two models for recency ranking into an existing multi-stage ranking architecture. Demonstrated an improvement in the effectiveness of the commercial search engine. \\
Technologies used: Python, Scala, Bash, Apache Spark, Hadoop.

\textbf{Microsoft, Cloud \& Enterprise}, Redmond, Washington, USA \vspace{1mm}\\\vspace{1mm}%
\textsl{Software Engineer Intern} \hfill September 2017 -- December 2017\\%
Performed data analysis and developed a visualization service for the existing log ingestion service in Azure. Worked on making the ingestion service more reliable, developed and fully integrated new monitoring features. \\
Technologies used: C\#, SQL, PowerShell, Azure, Git.

\textbf{Microsoft, Cloud \& Enterprise}, Redmond, Washington, USA \vspace{1mm}\\\vspace{1mm}%
\textsl{Software Engineer Intern} \hfill July 2016 -- September 2016\\%
Built a distributed, scalable microservice that uses Redfish, a new open industry hardware monitoring standard, runs on Service Fabric, and uses the Reliable Actors framework. The component was fully integrated in Azure Stack. \\
Technologies used: C\#, PowerShell, Service Fabric, Azure Stack, Git.

\textbf{Noom, Inc.}, New York, New York, USA \vspace{1mm}\\\vspace{1mm}%
\textsl{Software Engineer Intern} \hfill July 2015 -- October 2015\\%
Built a web application to improve meal logging experiences by clustering and processing users' food suggestions for multiple languages. Devised a new way of clustering existing data for easier processing. \\
Technologies used: Python, Flask, SQLAlchemy, Jinja2.

\textbf{X.FER}, Zagreb, Croatia \vspace{1mm}\\\vspace{1mm}%
\textsl{President, Problem Setter and Lecturer} \hfill October 2012 -- May 2017\\%
Presided X.FER, an informatics student association, and lead its main project, the course Competitive Programming. Also responsible for giving lectures, setting up homework assignments and exams designed to help students learn algorithms and their application in solving complex problems.

% \textbf{Board of European Students of Technology}, Zagreb, Croatia \vspace{2mm}\\\vspace{1mm}%
% \textsl{IT support} \hfill October 2013 -- current\\%
% Developing and maintaining websites using PHP, JavaScript, HTML5 and CSS3.

\textbf{Faculty of Electrical Engineering and Computing}, Zagreb, Croatia \vspace{1mm}\\\vspace{1mm}%
\textsl{Student Assistant} \hfill October 2012 -- February 2014\\%
Helping students with laboratory work and general understanding of topics covered in courses Algorithms  and  Data  Structures, Fundamentals  of  Electrical  Engineering and  Electronics 1.

%\vspace{1mm}

%\section{\mysidestyle Other experience}
%Lecturing high school competitors in programming:
%\begin{list2}
%\item V. gymnasium, Zagreb, Croatia
%\end{list2}

\vspace{1mm}
\section{\mysidestyle Publications}

\textbf{Debunking Sentiment Lexicons: A Case of Domain-Specific Sentiment Classification for Croatian} \vspace{1mm}\\\vspace{1mm}%
\textsl{Paper submission for BSNLP 2017} \hfill October 2016 -- February 2017\\%
Built a semi-supervised graph-based method to acquire sentiment lexicons from a corpus and experimented with acquisition parameters. Evaluated the lexicon-based models on the task of domain-specific sentiment classification and compared them against supervised models.
% Technologies used: Python, scikit-learn, Git.

\textbf{TakeLab at SemEval-2016: Using a Genetic Algorithm Based Ensemble} \vspace{1mm}\\\vspace{1mm}%
\textsl{Paper submission for Task 6: Stance Classification in Tweets.} \hfill October 2015 -- February 2016\\%
In a team of 9, built a system for the detection of stances in tweets. The system uses an ensemble of learning algorithms, classifiers, lexical and task-specific features, and is fine-tuned using a genetic algorithm. The system ranked 3rd among the 19 systems submitted to this task.
% Technologies used: Python, scikit-learn, Git.

\vspace{1mm}
\section{\mysidestyle Programming Competitions} 

Croatian Nationals in Informatics (DMIH) \hfill May 2006 -- May 2010\\%
Croatian Open Competition in Informatics (COCI) \hfill October 2006 -- April 2011%

\vspace{1mm}
\section{\mysidestyle Skills}

\begin{tabular}{@{}ll@{}}
Advanced & Python, Linux, Git, Bash \\%
Working knowledge & C, C\#, Java, Scala, Apache Spark, LaTeX \\%
Basic & Haskell, JavaScript \\%
\end{tabular}

%\vspace{1mm}
%\section{\mysidestyle Notable projects \& courses}

%\textbf{Bachelor Project: CROntroverza} \vspace{2mm}\\\vspace{1mm}%
%\textsl{A project from the world of Natural Language Processing.} \hfill October 2014 -- February 2015\\%
%Developed a system that analyzes Croatian news articles, clusters them into groups using the k-means clustering algorithm and determines the level of controversy by processing the comments. \\
%Technologies used: Haskell, Cabal, Git, Support Vector Machine.

%\textbf{Introduction to Java Programming Language} \vspace{2mm}\\\vspace{1mm}%
%\textsl{A course on Java and all things related.} \hfill March 2014 -- July 2014\\%
%Developed desktop, Web and Android applications, built a custom Paint and Web server, implemented the MVC pattern %on a blog system. For the final project, developed a desktop implementation of a multiplexer tree. \\
%Technologies used: Java, Swing, JSP, Apache Tomcat, Git, ANT, MySQL.

% \vspace{1mm}
% \section{\mysidestyle Hobbies}
% Travel, Languages, Fitness, Reading, Dogs

\end{resume}
\end{document}
